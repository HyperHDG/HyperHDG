\documentclass[a4paper, english, 12pt, reqno, draft]{amsart}

% ---------------------------------------------------------------------------

% Usepackages recommended in mcom-l-template.tex if needed.

\usepackage{amssymb}
% \usepackage{graphicx}
% \usepackage[cmtip,all]{xy}

% ---------------------------------------------------------------------------

% Usepackages inserted by authors.

\usepackage[english]{babel}
\usepackage[utf8]{inputenc}
\usepackage{tikz,tikzscale}
\tikzset{>=latex}
\usepackage{xcolor}
\usepackage{enumerate}
\usepackage{booktabs,multirow}
\usepackage{amsfonts}

%%%%%%%%%% Packages to help editing %%%%%%%%%%
\usepackage[notref,notcite]{showkeys} % Show labels.
\usepackage[mode=multiuser]{fixme} % Provide note, warning, error, fatal.
\FXRegisterAuthor{gk}{envgk}{GK}
\FXRegisterAuthor{ar}{envar}{AR}
\fxusetheme{color}

% ---------------------------------------------------------------------------

% Theorems in the AMS style.

\newtheorem{theorem}{Theorem}[section]
\newtheorem{lemma}[theorem]{Lemma}
\newtheorem{corollary}[theorem]{Corollary}

\theoremstyle{definition}
\newtheorem{definition}[theorem]{Definition}
\newtheorem{example}[theorem]{Example}
\newtheorem{exercise}[theorem]{Exercise}

\theoremstyle{remark}
\newtheorem{remark}[theorem]{Remark}

\numberwithin{equation}{section}

% ---------------------------------------------------------------------------

% Newcommands by the authors.

\newcommand{\graph}{\ensuremath{\mathcal G}}
\newcommand{\setEdge}{\ensuremath{\mathcal E}}
\newcommand{\setNode}{\ensuremath{\mathcal N}}
\newcommand{\edge}{\ensuremath{E}}
\newcommand{\node}{\ensuremath{N}}
\newcommand{\locDim}{\ensuremath{\mathfrak d}}
\newcommand{\globDim}{\ensuremath{\mathfrak D}}

\newcommand{\IN}{\ensuremath{\mathbb N}}
\newcommand{\IR}{\ensuremath{\mathbb R}}

\newcommand{\elem}{\ensuremath{T}}
\newcommand{\mesh}{\ensuremath{\mathcal T}}
\newcommand{\meshIndex}{\ensuremath{\mathcal I}}
\newcommand{\faceSet}{\ensuremath{\mathcal F}}
\newcommand{\faceSetDir}{\ensuremath{\mathcal F^\textup D}}
\newcommand{\face}{\ensuremath{F}}
\newcommand{\skeletal}{\ensuremath{\Sigma}}
\newcommand{\skeletalSpace}{\ensuremath{M}}
\newcommand{\skeletalSpaceHDG}{\ensuremath{M}}
\newcommand{\contElementSpace}{\ensuremath{V^\textup c}}
\newcommand{\contThreeElementSpace}[1]{\ensuremath{V^\textup c_{#1,p+3}}}
\newcommand{\linElementSpace}{\ensuremath{\overline V^\textup c}}
\newcommand{\discElementSpace}{\ensuremath{V}}
\newcommand{\polynomials}{\ensuremath{\mathcal P}}
\newcommand{\level}{\ensuremath{\ell}}
\newcommand{\iterMgOuter}{i}%\ensuremath{\nu}}
\newcommand{\iterMgInner}{m}%\ensuremath{\eta}}

\newcommand{\Div}{\nabla\!\cdot\!}
\newcommand{\discLaplacian}{\ensuremath{A}}
\newcommand{\extensionOp}{\ensuremath{\mathcal U^\textup c}}
\newcommand{\averagingOp}{\ensuremath{I^\textup{avg}}}
\newcommand{\restrictionOp}{\ensuremath{I^\textup{res}}}
\newcommand{\traceOp}{\ensuremath{\gamma_0}}
\newcommand{\injectionOp}{\ensuremath{I}}
\newcommand{\projectionOp}{\ensuremath{P}}
\newcommand{\projectionOrthogonalOP}{\ensuremath{\Pi}}
\newcommand{\projectionLinOp}{\ensuremath{\overline P}}
\newcommand{\skeletalProj}{\ensuremath{\Pi^\partial}}
\newcommand{\contLinProj}{\ensuremath{\overline \Pi^\textup c}}
\newcommand{\discProj}{\ensuremath{\Pi^\textup d}}
\newcommand{\liftingOp}{\ensuremath{S}}

\renewcommand{\vec}[1]{\ensuremath{\boldsymbol{#1}}}
\newcommand{\Nu}{\ensuremath{\vec \nu}}
\newcommand{\dx}{\ensuremath{\, \textup d x}}
\newcommand{\ds}{\ensuremath{\, \textup d \sigma}}
\newcommand{\localU}{\ensuremath{\mathcal U}}
\newcommand{\localQ}{\ensuremath{\vec{\mathcal Q}}}
\newcommand{\penaltyParam}{\ensuremath{\eta}}

\newcommand{\tildelambda}{\tilde\lambda}
\newcommand{\ureconstructed}{\overline{u}}
\newcommand{\projectionBramble}{\ensuremath{\bar B}}
\newcommand{\indexDOF}{\ensuremath{i}}
\newcommand{\IndexDOF}{\ensuremath{\mathcal I}}

\newcommand{\llangle}{\ensuremath{\langle \! \langle}}
\newcommand{\rrangle}{\ensuremath{\rangle \! \rangle}}
\newcommand{\nnorm}{\ensuremath{\vert \! \vert \! \vert}}

% ---------------------------------------------------------------------------

% Usepackage hyperref is the last to be added!

\usepackage[colorlinks = true, linkcolor = blue, citecolor = blue, urlcolor = blue]{hyperref}

% ---------------------------------------------------------------------------

% Document head as described in the template.

\begin{document}

\title{H\MakeLowercase{yper}HDG --- Hybrid disontinuous Galerkin methods for PDEs on hypergraphs} 

\author{Andreas Rupp}
\address{Interdisciplinary Center for Scientific Computing (IWR), Heidelberg University, Mathematikon, Im Neuenheimer Feld 205, 69120 Heidelberg, Germany}
% \curraddr{}
\email{andreas.rupp@fau.de, andreas.rupp@uni-heidelberg.de}
\thanks{This work is supported by the Deutsche Forschungsgemeinschaft (DFG, German Research Foundation) under Germany's Excellence Strategy EXC 2181/1 - 390900948 (the Heidelberg STRUCTURES Excellence Cluster).}

\author{Guido Kanschat}
\address{Interdisciplinary Center for Scientific Computing (IWR) and Mathematics Center Heidelberg (MATCH), Heidelberg University, Mathematikon, Im Neuenheimer Feld 205, 69120 Heidelberg, Germany}
% \curraddr{}
\email{kanschat@uni-heidelberg.de}
% \thanks{}

\subjclass[2010]{\textcolor{red}{TODO}}

\date{\today}

% \dedicatory{}

\begin{abstract}
 \textcolor{red}{\textsc{Todo!}} 
 \\[1ex] \noindent \textsc{Keywords.}
 \textcolor{red}{TODO!}
\end{abstract}
% 
\maketitle
% 
\section{Introduction}
% 
% ---------------------------------------------------------------------
\section{Hypergraphs}\label{SEC:hypergraph}
% ---------------------------------------------------------------------
% 
\subsection{Topologic aspects of hypergraphs}
% 
In the whole manuscript, we consider a \emph{hypergraph} $\graph = (\setNode,\setEdge)$ consisting of a finite set of \emph{hypernodes} $\setNode = \{\node_1, \node_2, \ldots \}$ and a set of \emph{hyperedges} $\setEdge = \{\edge_1, \edge_2, \ldots \}$. All hyperedges $\edge \in \setEdge$ can be interpreted as connecting several nodes and, thus, can be written as $\edge \subset \setNode$.

\paragraph{Cubic and simplicial hypergraphs and local dimension $\locDim$}
% 
We will define partial differential equations (PDEs) on hypergraphs. To do so, it helps---but is not necessary---to have an adequate graphical representation of hyperedges. For example, one might try to represent all $\edge \in \setEdge$ as open hypercubes of dimension $\locDim \in \IN$. A hypergraph, of which all hypereges $\edge = \{ \node_1, \node_2, \ldots, \node_{2\locDim} \}$ consist of exactly $2\locDim$ hypernodes, allows for such a representation of all hyperedges. The hypernodes may then be interpreted as the \emph{faces} (not the vertices) of the hypercubes/hyperedges and the hypergraph is denoted \emph{cubic} of \emph{local dimension} $\locDim$.

Analogously, a hypergraph of which all hyperedges consist of $\locDim+1$ is called simplicial of local dimension $\locDim$.

\paragraph{Cubic/siplicial embeddable hypergraphs and global dimension $\globDim$}
% 
If one wants to represent the whole hypergraph at once (instead of only single hyperedges), an embeddind to an $\globDim$ dimensional space is needed. A cubic/simplicial hypergraph can be embedded into $\IR^\globDim$ if there is a way to assemble all hypercubes/simplices (representing the hyperedges) such that for two hyperedges $\edge^+ \neq \edge^-$, we have $\edge^- \cap \edge^+ = \emptyset$.

If additionally for two hypernodes $\node^+ \neq \node^-$ the intersection of their interiors is empty, the hypergraph is denoted \emph{regularly embeddable} to dimension $\globDim$.

\paragraph{Examples for cubic hypergraphs}
% 
According to this definition, a graph is both a cubic and a simplicial hypergraph.

% ---------------------------------------------------------------------
\section{Model equation and discretization}\label{SEC:basics}
% ---------------------------------------------------------------------
% 
We consider the standard diffusion equation in mixed form defined on a polygonally bounded Lipschitz domain $\Omega \subset \mathbb R^d$ with boundary $\partial \Omega$. We assume homogeneous Dirichlet boundary conditions on $\partial\Omega$. Thus, we approximate solutions $(u, \vec q)$ of
% 
\begin{subequations}\label{EQ:diffusion_mixed}
\begin{align}
 \Div \vec q & = f && \text{ in } \Omega,\\
 \vec q + \nabla u & = 0 && \text{ in } \Omega,\\
 u & = 0 && \text{ on } \partial \Omega,%\\
%  \vec q \cdot \Nu & = g_\text N && \text{ on } \Gamma_\text N,
\end{align}
\end{subequations}
% 
for a given function $f$. Here, and in the following, $L^2(\Omega)$ denotes the space of square integrable functions on $\Omega$ with inner product and norm
\begin{equation}
 (u,v)_0 := \int_\Omega u v \dx, \qquad \text{and} \qquad \| u \|^2_0 := (u,u)_0.
\end{equation}
%
The space $H^k(\Omega)$ is the Sobolev space of $k$-times weakly
differentiable functions with derivatives in $L^2(\Omega)$.
% 
% ---------------------------------------------------------------------
\subsection{Spaces for the HDG on hypergraphs}
% ---------------------------------------------------------------------
%
Starting out from a subdivision $\graph = (\setEdge, \setNode)$.

By $\faceSet$ we denote the set of faces of $\mesh_\level$.
The subset of faces on the boundary is
\begin{gather}
  \faceSetDir_\level := \{\face \in \faceSet_\level : \face \subset \partial \Omega \}.
\end{gather}
Moreover, we define
$\faceSet^\elem_\level := \{ \face \in \faceSet_\level : \face \subset
\partial \elem \}$ as the set of faces of a cell $\elem\in\mesh_\level$.  On the set of faces, we define the space $L^2(\faceSet_\ell)$ as the space of
square integrable functions with the inner product
\begin{gather}
  \llangle \lambda, \mu \rrangle_\level = \sum_{\elem \in \mesh_\level} \int_{\partial \elem} \lambda\mu\ds,
\end{gather}
and its induced norm
$\nnorm \mu \nnorm^2_\level = \llangle \mu, \mu
\rrangle_\level$. Note that interior faces appear twice in this definition such that expressions like $\llangle u, \mu \rrangle_\level$ with possibly discontinuous $u|_{\elem} \in H^1(\elem)$ for all $\elem \in \mesh_\level$ and $\mu \in L^2(\faceSet)$ are defined without further ado. Additionally, we define an inner product
commensurate with the $L^2$-inner product in the bulk domain, namely
\begin{gather}
  \langle \lambda, \mu \rangle_\level
  = \sum_{\elem \in \mesh_\level} \frac{|\elem|}{|\partial \elem|}
  \int_{\partial \elem} \lambda \mu \ds \cong \sum_{\face \in \faceSet_\level} h_\face
  \int_{\face} \lambda \mu \ds.
\end{gather}
Its induced norm is $ \| \mu \|^2_\level = \langle \mu, \mu \rangle_\level$.

Let $p\ge 1$ and $\polynomials_p$ be the space of (multivariate)
polynomials of degree up to $p$. Then, we define the space of piecewise
polynomials on the skeleton by
\begin{gather}
  \label{EQ:skeletal_space}
  \skeletalSpace_\level := \left\{ \lambda \in L^2(\faceSet_\level) \;\middle|\;
    \begin{array}{r@{\,}c@{\,}ll}
  \lambda_{|\face} &\in& \polynomials_p & \forall \face \in \faceSet_\level\\
  \lambda_{|\face} &=& 0 & \forall \face \in \faceSetDir_\level    
    \end{array}
  \right\}.
\end{gather}

The HDG method involves a local solver on each mesh cell
$\elem \in \mesh_\level$, producing cellwise approximations $u_\elem \in V_\elem$
and and $\vec q_\elem\in \vec W_\elem$ of the functions $u$ and $\vec q$ in
equation~\eqref{EQ:diffusion_mixed}, respectively. We choose
$V_\elem = \polynomials_p$. Then, choosing
$\vec W_\elem = \polynomials_p^d$ yields the so called hybridizable local discontinuous Galerkin (LDG-H) scheme. Our current analysis is in fact limited to this case and other choices require a modification of Lemma~\ref{LEM:u_q_bound}. We will also use the concatenations of the spaces $V_\elem$
and $\vec W_\elem$, respectively, as a function space on $\Omega$, namely
\begin{gather}
  \label{EQ:dg_spaces}
  \begin{aligned}
    \discElementSpace_\level
    &:=\bigl\{ v \in L^2(\Omega)
    & \big|\;v_{|\elem} &\in V_\elem,
    &\forall \elem &\in \mesh_\level \bigr\},\\
    \vec W_\level
    &:=\bigl\{ \vec q \in L^2(\Omega;\mathbb R^d)
    & \big|\;\vec q_{|\elem} &\in \vec W_\elem,
    &\forall \elem &\in \mesh_\level \bigr\}.    
  \end{aligned}
\end{gather}
%\begin{alignat}2
% \end{alignat}



% 
\bibliographystyle{alpha}
\bibliography{HyperHDG}
% 
\end{document}
